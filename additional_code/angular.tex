\documentclass[10pt]{amsart}

\usepackage{gensymb}
\usepackage{listings}

\numberwithin{equation}{subsection}

\newcommand{\nuc}[2]{${}^{#1}\textrm{#2}$}
\newcommand{\mnuc}[2]{{}^{#1}\textrm{#2}}
\newcommand{\react}[4]{$#1(#2,#3)#4$}
\newcommand{\mreact}[4]{#1(#2,#3)#4}
\newcommand{\alpa}{\react{\mnuc{27}{Al}}{\textrm{p}}{\alpha}{\mnuc{24}{Mg}}}

\title{Angular Correlations}
\author{Mike Moran}
\date{\today}

\begin{document}
\maketitle

If we have an initial state with particles $a$ and $A$ with a channel spin of
$(j^{\pi})_1$, and the lighter particle $a$ has an orbital angular momentum of
$\ell_a$, that creates a state $J^{\pi}$ in a compound nucleus $C$ that then
decays into a second two-body system with particles $b$ and $B$, where the new
channel spin in $(j^{\pi})_2$ and the lighter particle $b$ has an orbital
angular momentum $\ell_b$, we can determine the angular correlation between the
two light particles $a$ and $b$ in the center of mass frame.

To do this, we need to first look at the coupling between the various quantum
states involved to see what possible quantum values there are for the orbital
angular momenta. Note that, in most cases, only the two lowest values of
$\ell_i$ are necessary, since higher values are drastically less likely to have
an impact on the reaction. Following that, we can determine the contributions
to the angular correlation arising from the mixed states (if they are present)
and relate that to our differential cross section.

For the remainder of this discussion, I will be using the \alpa{} reaction and
looking at two states identified in Andersen~\textit{et al.},~1961: the
isotropic $E_p = 1182$~keV state with $J^{\pi} = 2^+$, and the non-isotropic
$E_p = 1315$~keV state with $J^{\pi} = 4^+$. The first of these states was
measured during the February 2017 experiment, and the second state would not be
reasonably detectable at $0\degree$.


\section{Quantum Number Selection}

Our quantum numbers are defined by the properties of the entrance and exit
channels and the state within the compount nucleus we are trying to populate.
We will approach these numbers by looking at them moving from the entrance
channel to the compound nucleus to the exit channel. We are assuming here that
all particles involved are nuclear particles and that there are no intervening
$\gamma$s to consider.

\subsection{Entrance Channel}

The entrance channel has a spin-parity of $(j^{\pi})_1$, determined by the spin
and parity of the two separate particles. The channel spin is determined by
\[
    |j_a - j_A| \leq j_1 \leq j_a + j_A,
\]
and the channel parity is determined by
\[
    \pi_1 = \pi_a\pi_A(-1)^{\ell_a}.
\]
These equations also hold for the exit channel, after making the replacements
$1\rightarrow2$, $a\rightarrow b$, and $A\rightarrow B$. The value of the
relative angular momentum, taken here to the orbital angular momentum of just
the smaller particle, will be constrained by the compound nuclear state, but
for now it can be considered a free parameter.

For our reaction, the proton has $J^{\pi} = 1/2^+$ and \nuc{27}{Al} has
$J^{\pi} = 5/2^+$, leading to two possible channel spins:
$j_1 = 2$ or 3. Again, the parity cannot be determined, since we have not
defined our orbital angular momentum.


\subsection{Compound Nucleus}

Within our compound nucleus, we select a single state with spin-parity
$J^{\pi}$ to populate. The energy kinematics would determine what that level's
excitation energy is and thus what the required beam energy would be to
populate the state, but that is not discussed here. Now that we know our
intermediate state's properties, we can determine the final possible values
for our entrance channel.

First, the entrance channel's $\ell_a$ can be set by following the triangular
rule
\[
    |J - j_1| \leq \ell_a \leq J + j_1.
\]
Since we also have the parity of our composite state, we can determine whether
our $\ell_a$ values will be odd or even, since
\[
    \pi = \pi_a\pi_A(-1)^{\ell_a},
\]
leading to the chain of possible values as $\ell, \ell + 2, \dotsb, \ell + n$.
As mentioned previous, we usually only have to worry about the two lowest
angular momenta, $\ell$ and $\ell + 2$, but in reality all would be considered.
Again, these equations also hold for the exit channel, after making the same
replacements as before.

For our example, the results of figuring out these possibilities are summarized
in below. Note that all $\ell_{\textrm{p}}$ values are even, since our entrance
channel and compound nuclear state parities are even.

\begin{description}
    \item[$J^{\pi} = 2^+$]
        $j_1 = 2 \rightarrow \ell_{\textrm{p}} = 0\textrm{, }2\textrm{, and }4$,
        $j_1 = 3 \rightarrow \ell_{\textrm{p}} = 2\textrm{ and }4$
    \item[$J^{\pi} = 4^+$]
        $j_1 = 2 \rightarrow \ell_{\textrm{p}} = 2\textrm{, }4\textrm{, and }6$,
        $j_1 = 3 \rightarrow \ell_{\textrm{p}} = 2\textrm{, }4\textrm{, and }6$
\end{description}


\subsection{Exit Channel}

Our exit channel properties are determined the same way as our exit channel:
finding the allowed $j_2$ based on the exit channel's two particles,
determining the allowed $\ell_b$ values by looking at the triangle rule, and
selecting states were the parities of the compound nuclear state and the exit
channel are the same. From here, we have assigned all possible quantum numbers
describing our reaction.

Our exit channel consists of our $\alpha$ particle and \nuc{24}{Mg} in its
ground state, both of which are $J^{\pi} = 0^+$. If we were instead looking at
the $\alpha_1$ channel, populating the first excited state in \nuc{24}{Mg}, we
would use the spin-parity ($2^+$) of that state and would need to handle the
outgoing $\gamma$ accordingly. In our case, our exit channel can only be
$(j^{\pi})_2 = 0^+$, which simplifies things greatly, by only allowing our
$\alpha$ particle to have $\ell_{\alpha} = 2$ (for our $2^+$ state) or
$\ell_{\alpha} = 4$ (for our $4^+$ state).


\subsection{Final Description}

To condense our qunatum number representation down, I will introduce the
following notation describing the sequence of angular momenta:
\[
    j_1(L_1)J(L_2)j_2,
\]
where we can make the substitution $L_i = \ell_{a|b}$ since we are assuming that
all of the orbital angular momentum is contained within the smaller particle
within each channel. For our two example resonances, we can write our allowed
quantum number sequences as
\[
    2\binom{0}{2}2(2)0\textrm{ and }3\binom{2}{4}2(2)0
\]
for our $J^{\pi} = 2^+$ resonance (note that we have limited ourselves to just
the two lowest $\ell_{\textrm{p}}$ numbers), and
\[
    2\binom{2}{4}4(4)0\textrm{ and }3\binom{2}{4}4(4)0
\]
for our $J^{\pi} = 4^+$ resonance. Now that we have our allowed quantum
numbers, we can move on to actually calculating the angular distributions.


\section{Angular Correlation}

The angular correlation between two detected particles, one incoming and one
outgoing, can be expressed as a sum of Legendre polynomials over the angle
between the two particles, given by
\begin{equation}
    \label{eq:angcorr}
    W(\theta) = \frac{1}{b_0}\sum_{n=0}^{n_{\textrm{max}}} b_nP_n(\cos\theta),
\end{equation}
where $P_n(x)$ are the Legendre polynomials. Due to symmetry concerns, we only
have to worry about even-$n$ terms within the sum, which reduces this down to
fewer terms. The upper limit $n_{\textrm{max}}$ is based on the specific values
of the various angular momenta involved, and the coefficients $b_n$ will be
discussed later. This distribution can be related to the differential and total
cross section through
\[
    \left(\frac{d\sigma}{d\Omega}\right)_{\theta} =
        \frac{1}{4\pi}\sigma W(\theta).
\]
If we have an isotopic angular distribution, which is the current status for
our two resonances, we must have $W(\theta) = 1$.

The coefficients $b_n$ are also based on the angular momentum states involved,
which include terms describing the vector coupling between the various states.
We will use the coefficients $F_n$ defined in Biedenharn, 1960:
\[
    F_n(LL'jJ) = (-1)^{j - J - 1}\sqrt{(2L + 1)(2L' + 1)(2J + 1)}
        \langle L1L'{-1}|n0\rangle W(JJLL';nj),
\]
where $\langle L_1m_1L_2m_2|JM\rangle$ are Clebsch-Gordan coefficients, and
$W(j_1j_2j_3j_4;j_5j_6)$ are Racah coefficients. These coefficients can either
be found in a tabulated resource or calculated. For our expansion coefficients
$b_n$, we'll have an $F_n$ for the entrance and exit channels.

Since our entrance channel has two possible spin states, we will have to
consider the mixing between those two states. The total angular correlation,
taking that mixing into account, is given by
\[
    W(\theta) = W_{j_s}(\theta) + \delta_c^2W_{j_s'}(\theta),
\]
where $\delta_c^2 = P_{j_s'} / P_{j_s}$ is the channel spin mixing ratio,
defined as the ratio between the probabilities for forming or decaying from the
intermediate state through the two spins. To note, we take $j_s' > j_s$, which
is the standard convention.

At this point, we have all of our scaffolding in place, so we just need to
bring all of the pieces together. Since our entrance channel can form the
compound state through two different orbital angular momenta $\ell_1$, we also
need to consider coherent interference between those two possibilities.

For a simple two-step process, our expansion coefficients are given by
\[
    b_n = a_n(1)A_n(1)a_n(2)A_n(2),
\]
where the factors $a_n(i)$ and $A_n(i)$ are determined by the angular momentum
of the entrance and exit channels. In our case, our entrance channel has
coherent interference between the various orbital angular momentum
possibilities, while our exit channel progresses through pure radiation. I will
discuss the simpler case first.


\subsection{Pure Radiation}

In a pure radiation, where only a single orbital angular momentum takes part,
there are no alterations to the coefficients, and they are only determined by
the properties of the particles involved. For $s = 0$ particles, which includes
our $\alpha$ particle, the coefficients are
\[
    a_n(i) = \frac{2L_i(L_i + 1)}{2L_i(L_i + 1) - n(n + 1)}
\]
and
\[
    A_n(i) = F_n(L_ij_iJ) = F_n(L_iL_ij_iJ).
\]
If we had $s \neq 0$ particles, we would need to use the possible channel spin
quantum numbers instead of just the initial state spin, something that we have
already taken into account. The sum over the Legendre polynomials given in
(\ref{eq:angcorr}) is restricted to $0 \leq n \leq \min(2L_1, 2L_2, 2J)$.


\subsection{Mixed Radiation}

For a channel where multiple orbital angular momenta take part in the formation
or decay from a composite state, we have an orbital angular momentum mixing
ratio $\delta_i^2 = \Gamma_{\ell_1} / \Gamma_{\ell_2}$, where $\Gamma_{\ell}$
being the particle partial width for a given orbital angular momentum. For
these channels, the coefficients $a_n(i)A_n(i)$ are replaced with
\[
    a_n(L_iL_i)F_n(L_ij_iJ) + 2\delta_ia_n(L_iL_i')F_n(L_iL_i'j_iJ) +
        \delta_i^2a_n(L_i'L_i')F_n(L_i'j_iJ),
\]
where we are assuming that $L_i' > L_i$ and, as mentioned before, we use the
channel spin $j_i$ instead of the state spin.

The values of $a_n(i)$ are also changed to represent the mixing between the
orbital angular momenta. Instead of the definition used previously, we use
\begin{align*}
    a_n(L_iL_i') &= \cos(\zeta_{L_i} - \zeta_{L_i'})
        \frac{\langle L_i0L_i'0|n0\rangle}{\langle L_i1L_i'{-1}|n0\rangle} \\
    &= \cos(\zeta_{L_i} - \zeta_{L_i'})
        \frac{2\sqrt{L_iL_i'(L_i + 1)(L_i' + 1)}}
        {L_i(L_i + 1) + L_i'(L_i' + 1) - n(n + 1)}.
\end{align*}
The charge particle phase shifts $\zeta_L$ are given by
\[
    \zeta_L = {-\arctan}\left(\frac{F_L}{G_L}\right) +
        \sum_{n = 1}^L\arctan\left(\frac{\eta}{n}\right),
\]
where $F_L$ and $G_L$ are the regular and irregular Coulomb wave functions,
respectively, and $\eta = Z_{a|b}Z_{A|B}e^2/(\hbar v)$ is the Sommerfeld
parameter. The Coulomb wave functions can either be found in a tabulation or
calculated using a program.


\subsection{Combining Everything}

Since the full equation for our reaction, combining all of the above pieces,
would easily fill a half page and not be any more intelligible than the above
breakdown, I will not lay out the entire thing here. Instead, I will describe
how the various pieces fit together.

Looking at both resonances, our \alpa{} reaction consists of an entrance
channel that exhibits mixing between orbital angular momentum states and
between channel spin states, and an exit channel that progresses through a
single pure radiation channel. We can write out each part by hand if we
desired, but since there are a large number of coefficients that we would need
to calculate anyway, it is easier to do the entire calculation through code.

Since the Coulomb wave functions are radius-dependent, we can also leave the
$\cos(\zeta_{L_i} - \zeta_{L_i'})$ term unevaluated. In addition, since higher
orbital angular momentum terms decrease the penetration factor for the
reaction, we could even simplify the orbital angular momentum mixing to not be
included, meaning that the angular correlation would, in our case, just be
based on the two different entrance channel spin states.


\section{Code}

Since we need to calculate the various coefficients in order to determine if
there is an angular correlation between the particles, and to see if fitted
experimental data matches up with the theory, we will use a short piece of code
to perform those calculations. The code below is written in \texttt{python}. In
addition, a separate package \verb+angular+ was written, making use of
additional mathematical packages: \verb+mpmath+, \verb+numpy+, and
\verb+sympy+, to perform all of the required coefficient calculations.

The code in place gives the coefficients present for each Legendre polynomial,
as further expansion can be handled by the end user. In addition, the
coefficients being displayed separately allow for an easier relation to
previous experimental results.

\vspace{1.5em}
\begin{lstlisting}[language=Python]
import angular

# check against Iliadis - should give [1, 2]
coeff = angular.angular_correlation(0, 1, 1, 1, 0)
print(coeff)
\end{lstlisting}


\end{document}
